\chapter{Installation}

\section{System requirements and portability}

HMMER is designed to run on UNIX platforms. The code is
POSIX-compliant ANSI C.  You need a UNIX machine and an ANSI C
compiler to build and use it.

Some optional tests and utilities use Perl, so having Perl installed
would be helpful.

I expect HMMER could be ported easily to other platforms. All POSIX
system calls in HMMER are optional, and can be safely removed in a
port to a non POSIX OS with a minimal loss of functionality. 

HMMER 2 has not yet been ported to other operating systems.  HMMER 1
was ported by other people to Digital VAX/VMS, Apple MacOS, Win95, and
WinNT with relatively little difficulty, and I've made efforts to
improve the portability of the HMMER 2 code.

I would greatly appreciate receiving diffs for any ports of HMMER to
any platform.

\section{Installing a precompiled distribution}

Precompiled binary distributions of HMMER are available for at least
Intel/Linux, Silicon Graphics IRIX 5.x and 6.x, and Sun Sparc Solaris
2.x platforms.

To install a precompiled distribution, do the following:

\begin{enumerate}
\item Download the distribution from http://hmmer.wustl.edu/.

\item Uncompress and un-tar it; for example, \\
	\user{uncompress hmmer-intel-linux.tar.Z}
	\user{tar xf hmmer-intel-linux.tar}

	A new directory \prog{hmmer-xx} is created, where ``xx'' is
	the HMMER version number.

\item Edit the top of the Makefile. 

To build and test HMMER in the distribution directory, you don't need
to edit the Makefile at all. But to permanently install HMMER on your
system, set the make variables BINDIR and MANDIR to be the directories
where you want HMMER executables and man pages to be installed. If you
are installing in directories called bin/ and man/ in your home
directory, you don't need to change anything.

\item (Optional) Type 'make test' to compile and run a test suite.
Some of the tests require that you have Perl installed. None of these
tests should fail.

\item Type 'make install' to install the programs and man pages. 
You may have to become root, depending on where you're installing.

\item Type 'make clean' to clean up.
\end{enumerate}

\section{Compiling from a source-only distribution}

\begin{enumerate}
\item Download the distribution from http://hmmer.wustl.edu/.

\item Uncompress and un-tar it:\\
	\user{uncompress hmmer.tar.Z}
	\user{tar xf hmmer.tar}

	A new directory \prog{hmmer-xx} is created, where ``xx'' is
	the HMMER version number.

\item Edit the top of the Makefile. 

To build and test HMMER in its source directory, you don't need to
edit the Makefile at all. 

To permanently install HMMER on your system, set the make variables
BINDIR and MANDIR to be the directories where you want HMMER
executables and man pages to be installed. If you are installing the
programs in /usr/local/bin and the man pages in /usr/local/man/man1,
you don't need to change anything.

The default Makefile is configured for cc (your vendor's compiler) and
-O optimization. The package is known to build "out of the box" on SGI
IRIX, Sun Solaris, GNU/Linux, or DEC Digital Unix platforms without
any special modifications. 

On SunOS 4.1.x systems, you will have to use the GNU gcc compiler,
because SunOS cc is not ANSI-compliant.

You can play with the compiler options in CFLAGS to try to get more
speed, if you're compiler-fluent. 

\item Type 'make' to build the programs.

\item (Optional) Type 'make test' to compile and run a test suite.
Some of the tests require that you have Perl installed. None of these
tests should fail.

\item Type 'make install' to install the programs and man pages. 
You may have to become root, depending on where you're installing.

\item Type 'make clean' to clean up. 
\end{enumerate}


\section{Environment variable configuration}

These steps are optional, and will only apply (I think) to
sufficiently POSIX-compliant operating systems like UNIX, Linux, or
WinNT.

HMMER reads three environment variables, and is designed to coexist
peacefully with an installation of WUBLAST or NCBI BLAST:

\begin{wideitem}
\item[\emprog{HMMERDB}] - directory location of HMM databases (e.g. PFAM)
\item[\emprog{BLASTDB}] - directory location of FASTA-formatted sequence databases
\item[\emprog{BLASTMAT}] - directory location of BLAST substitution matrices
\end{wideitem}

If you have installed BLAST, you probably already have these 
environment variables set in system-wide or user-specific
.cshrc files. They are optional. If they are set up, you
can simplify command lines to:

\vspace{1.5em}
\user{hmmpfam pfam my.query}
\user{hmmsearch my.hmm swiss35}

instead of

\vspace{1.5em}
\user{hmmpfam   /some/long/path/to/databases/pfam my.query}
\user{hmmsearch my.hmm /some/long/path/to/databases/swiss35}

\section{Recommended systems}

HMMER is currently developed and maintained on the following three
systems:

\begin{itemize}
\item Silicon Graphics Origin200 dual R10K server, IRIX 6.4
\item NEC Versa 6050MX laptop, Intel Pentium, 
      Red Hat Linux 4.2, configured by VA/Research
\item Dell Dimension desktop, Intel PentiumII, Red Hat
      Linux 4.2
\end{itemize}

At each release, HMMER is also tested on the following systems:

\begin{itemize}
\item Sun UltraSparc 1, Solaris 5.5
\end{itemize}

\section{Make targets}

There are a variety of make targets in the toplevel Makefile.  For
completeness, they are summarized here. \textit{You shouldn't need to
know this stuff.} Targets marked (Private) are targets which are not
supported in the public HMMER release. They only run in my local
development environment.

\begin{wideitem}
\item[\textbf{all}]  Compiles the source code; puts the compiled
programs in the binaries/ subdirectory.
 
\item[\textbf{test}]  Compiles and runs the Shiva testsuite.

\item[\textbf{install}] Installs the programs in BINDIR and
the man pages in MANDIR.

\item[\textbf{clean}] Cleans up the directory, leaving the
distribution files.

\item[\textbf{verify}] Runs consistency checks on the package.
(Private)

\item[\textbf{doc}] Builds the Userguide from \latex source. (Private)

\item[\textbf{dist}] Checks out a complete distribution from RCS
and assembles the .tar.Z file. (Private)

\item[\textbf{ftpdist}] Installs a new release on the FTP
site. (Private)

\item[\textbf{stable}] Symlinks a new FTP release to hmmer.tar.Z. (Private)

\item[\textbf{almostclean}] Like 'make clean', but leaves the
binaries/ subdirectory. Used for making an executable distribution.
(Private)
\end{wideitem}







