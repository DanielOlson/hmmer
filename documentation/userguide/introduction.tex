\section{Introduction}
\setcounter{footnote}{0}

HMMER is for searching sequence databases for homologs of protein or
DNA sequences, and also for making large multiple sequence alignments.
HMMER makes a statistical \emph{profile} of a multiple sequence
alignment (or even a single sequence) that you provide. This profile
assigns a position-specific scoring system for substitutions,
insertions, and deletions, using probabilistic models called ``profile
hidden Markov models'' (profile HMMs) \citep{Krogh94,Eddy98,Durbin98}.
HMMER provides tools for searching sequence databases with profiles,
searching profile databases with sequences, and aligning sequences to
profiles. Although the first implementations of profile HMMs were very
computationally expensive, HMMER is now typically faster than BLASTP
or FASTA searches\citep{Eddy11}.\footnote{BLASTN remains two orders of
  magnitude faster than HMMER DNA searches, but is far less
  sensitive.}

\subsection{How to avoid reading this manual}

We hate reading documentation. If you're like us, you're thinking,
\pageref*{manualend} pages of documentation, you're joking! I just
want to know that the software compiles, runs, and gives apparently
useful results, before I read some \pageref*{manualend} exhausting
pages of someone's documentation. For cynics who have seen one too
many software packages that don't work:

\begin{itemize}

\item Follow the quick installation instructions on page
  \pageref{section:installation}. An automated test suite is included,
  so you will know immediately if something went
  wrong.\footnote{Nothing should go wrong.}

\item Go to the tutorial section on page \pageref{section:tutorial},
  which walks you through some examples of using HMMER on real data.

\end{itemize}

Everything else, you can come back and read later.



\subsection{How to avoid using this software (links to similar software)}

Other implementations of profile HMM methods and position-specific
scoring matrix methods are available, including:

\begin{center}
\begin{tabular}{lp{5in}l}
Software  &   URL \\ \hline
HH-SUITE  & \url{http://www.soeding.genzentrum.lmu.de/software-and-servers-2/}\\
PSI-BLAST & \url{https://blast.ncbi.nlm.nih.gov/}\\
PFTOOLS   & \url{http://web.expasy.org/pftools/}\\
SAM       & \url{https://compbio.soe.ucsc.edu/sam.html}\\
\end{tabular}
\end{center}




\subsection{What profile HMMs are}

Profile HMMs are statistical models of multiple sequence alignments,
or even of single sequences. They capture position-specific
information about how conserved each column of the alignment is, and
which symbols are likely. Anders Krogh, David Haussler, and co-workers
at UC Santa Cruz introduced profile HMMs to computational biology
\citep{Krogh94}, adopting HMM techniques which have been used for
years in speech recognition.  HMMs had been used in biology before the
Krogh/Haussler work, notably by Gary Churchill \citep{Churchill89},
but the Krogh paper had a dramatic impact because HMM technology was
so well suited to ``profile'' methods that were already popular.

Profiles had been introduced earlier by Gribskov and colleagues
\citep{Gribskov87,Gribskov90}, and several other groups introduced
similar approaches at about the same time, including ``flexible
patterns'' \citep{Barton90} and
``templates''\citep{Bashford87,Taylor86}. The term ``profile'' has
stuck.\footnote{Another popular term is a ``position specific scoring
  matrix'', a PSSM, but some small nocturnal North American marsupials
  have a prior claim on this name.}  All profile methods (including
PSI-BLAST \citep{Altschul97}) are more or less statistical
descriptions of the consensus of a multiple sequence alignment. They
use \emph{position-specific} scores for amino acids or nucleotides
(residues) and position specific penalties for opening and extending
an insertion or deletion.  Traditional pairwise alignment (for
example, BLAST \citep{Altschul90}, FASTA \citep{Pearson88}, or the
Smith/Waterman algorithm \citep{Smith81}) uses position-{\em
  independent} scoring parameters. Position-specific scoring captures
important information about the degree of conservation at various
positions in the multiple alignment, and the varying degree to which
gaps and insertions are permitted.

An advantage of using HMMs is that HMMs have a formal probabilistic
basis, allowing us to use probability theory to set the large number
of free parameters in a profile. Though this might sound like an
academic issue, a probabilistic basis lets us do things that more
heuristic methods cannot do easily. One of the most important is that
HMMs have a consistent theory for setting position-specific gap and
insertion scores. The methods are consistent and therefore
automatable, allowing us to make libraries of hundreds of profile HMMs
and apply them on a large scale to whole genome analysis.  One such
database of protein domain models is Pfam
\citep{Sonnhammer97,FinnBateman16}, which is part of the Interpro
protein domain annotation system \citep{MitchellFinn15}. The
construction and use of Pfam is tightly tied to the HMMER software
package.

Profile HMMs do have important limitations. One is that HMMs do not
capture any higher-order correlations.  An HMM assumes that the
identity of a particular position is independent of the identity of
all other positions.\footnote{This is not strictly true. There is a
  subtle difference between an HMM's state path (a first order Markov
  chain) and the sequence described by an HMM (generated from the
  state path by independent emissions of symbols at each state).}
Profile HMMs are often not good models of structural RNAs, for
instance, because an HMM cannot describe base pairs.



\subsection{Applications of profile HMMs}

A principal application of HMMER is when you are working on a sequence
family, and you have carefully constructed a representative multiple
sequence alignment. Your family, like most protein (or DNA/RNA)
families, has a number of strongly conserved key amino acids (or
nucleotides), separated by characteristic spacings. You wonder if
there are more members of your family in the sequence databases, but
the family is so evolutionarily diverse, a BLAST search with any
individual sequence doesn't even find the rest of the sequences you
already know about. You're sure there are some distantly related
sequences in the noise. You spend many pleasant evenings scanning weak
BLAST alignments by eye to find ones with the right key residues are
in the right places, but you sure wish there was a computer program
that did this a little better. This is the main problem HMMER was
designed to solve.

You don't have to have a multiple alignment; you can search even with
single sequences. HMMER includes two programs for searching protein
databases with single query sequences: \prog{phmmer} and
\prog{jackhmmer}, where \prog{phmmer} aims to supplant BLASTP, and
\prog{jackhmmer} aims to supplant PSI-BLAST.

Another application is the automated annotation of the domain
structure of proteins. Large databases of curated alignments and HMMER
models of known domains are available, including Pfam
\citep{FinnBateman16} and SMART \citep{Letunic12} in the Interpro
database consortium \citep{MitchellFinn15}. Many ``top ten protein
domains'' lists, a \emph{de rigueur} table in genome analysis papers,
rely heavily on HMMER annotation.  Say you have a new sequence that,
according to a BLAST analysis, shows a slew of hits to receptor
tyrosine kinases. Before you decide to call your sequence an RTK
homologue, you suspiciously recall that RTK's are, like many proteins,
composed of multiple functional domains, and these domains are often
found promiscuously in proteins with a wide variety of functions. Is
your sequence really an RTK? Or is it a novel sequence that just
happens to have a protein kinase catalytic domain or fibronectin type
III domain?

Another application is the automated construction and maintenance of
large multiple alignment databases.  It is useful to organize
sequences into evolutionarily related families. But there are
thousands of sequence families, some of which comprise tens of
thousands of sequences -- and the primary sequence databases continue
to double every year or two. This is a hopeless task for manual
curation; but on the other hand, manual curation is still necessary
for high-quality, biologically relevant multiple alignments. Databases
like Pfam \citep{FinnBateman16} and Dfam \citep{Wheeler13} are
constructed by distinguishing between a stable curated ``seed''
alignment of a small number of representative sequences, and ``full''
alignments of all detectable homologs. HMMER is used to make a model
of the seed, search the database for homologs, and can automatically
produce the full alignment by aligning every sequence to the seed
consensus.

You may find other applications as well. Using hidden Markov models to
make a linear consensus model of a bunch of related strings is a
pretty generic problem, and not just in biological sequence analysis.
HMMER code has even been hacked to model music, polymorphic computer
viruses, speech, mouse songs, and even automobile engine
telemetry.\footnote{True. I told an automotive engineer he had
  permission, provided I never see one of my own error messages
  show up on my Honda's dashboard.}  If you use it for something
particularly strange, we'd be curious to hear about it.


\subsection{HMMER uses ensemble algorithms, not optimal alignment}

Historically, most sequence search tools look for optimal (or
high-scoring) alignments. However, sequence alignments are uncertain,
and the more distantly related sequences are, the more uncertain
alignments are. Instead of using optimal alignment algorithms, HMMER
uses ensemble algorithms that sum probabilities over all possible
alignments.\footnote{In HMM-speak, we use the Forward algorithm (and
  variants of it), not the Viterbi algorithm.} This is one reason that
HMMER gets more power than tools that depend on optimal alignments.

The use of ensemble algorithms shows up in several features that
differentiate HMMER from older methods:

\paragraph{Explicit representation of alignment uncertainty.}
  When HMMER shows an alignment, it also calculates how much
  probability mass that one alignment has in the overall ensemble --
  which means HMMER can annotate a probabilistic confidence in an
  alignment, or in each individual aligned residue. Some downstream
  analyses that depend on alignments (such as phylogenetic tree
  inference) benefit from being able to distinguish confidently
  aligned from not confidently aligned residues.

\paragraph{Sequence scores, not alignment scores.} HMMER's
  log-odds scores for a sequence aren't just optimal alignment scores;
  they are summed over the posterior alignment ensemble. Statistical
  inference theory says that scores based on a single optimal
  alignment are an approximation that breaks down when alignments are
  uncertain; HMMER's calculation is the full, unapproximated
  calculation.
 
\paragraph{Speed.} The ensemble (Forward) algorithm is more
  computationally intensive than optimal alignment algorithms.  HMMER
  implements heuristic acceleration of the HMM Forward algorithm using
  vectorization technology available on modern processors. As
  vectorization technology continues to advance, HMMER searches are
  now typically faster than BLAST.\\


Individually, none of these points is new. As far as alignment
ensembles and sequence scores go, one main reason why hidden Markov
models were so theoretically attractive in the first place for
sequence analysis is that they are good probabilistic models for
explicitly dealing with alignment uncertainty. The SAM profile HMM
software from UC Santa Cruz has always used full probabilistic
inference (the HMM Forward/Backward algorithms) as opposed to optimal
alignment scores (the HMM Viterbi algorithm). HMMER2 had the full HMM
inference algorithms available as command-line options, but used
Viterbi optimal alignment by default, in part for speed
reasons.

One reason why it's been hard to deploy sequence scores for practical
large-scale use is that we haven't known how to accurately calculate
the statistical significance of a log-odds score that's been summed
over alignment uncertainty. Accurate statistical significance
estimates are essential when one is trying to discriminate homologs
from millions of unrelated sequences in a large sequence database
search. The statistical significance of optimal local alignment scores
is calculated by Karlin/Altschul statistics
\citep{Karlin90,KarlinAltschul93}. Karlin/Altschul statistics are one
of the most important and fundamental advances introduced by BLAST.
However, this theory doesn't apply to HMMER's ensemble log-odds
sequence scores (HMM ``Forward scores''). The statistical significance
(expectation values, or E-values) of HMMER sequence scores is
determined by using a theoretical conjecture about the statistical
properties of ensemble log-odds scores which have been supported by
numerical simulation experiments \citep{Eddy08}.

And as far as speed goes, the pioneers of heuristic acceleration of
sequence database searches are the folks behind BLAST and FASTA, who
developed effective heuristics that closely approximate an
unaccelerated Smith/Waterman dynamic programming search.  The first
implementations of profile HMM methods used dynamic programming
without heuristics (the profile HMM Viterbi algorithm is essentially
Smith/Waterman with position-specific scoring), so they were more
comparable in speed to Smith/Waterman than to BLAST; plus, using the
Forward algorithm slowed them down still more. It was a while before
we invested the time to develop heuristic acceleration of profile HMM
methods. Our design goal in HMMER3, released in 2010, was to achieve
rough speed parity with BLAST and FASTA. We did have to develop a new
and more sensitive heuristic, because we wanted to retain the power of
the ensemble sequence scores, and it happens that what we did is
suited to modern hardware architectures, so we benefit
disproportionally from advances in commodity processors. Lately, this
means that HMMER's speed has started to pull away from BLAST and
FASTA.



\subsection{What's missing}

Some of the more important holes for us are:

\paragraph{Translated comparisons.} We'd of course love to have the HMM
equivalents of BLASTX, TBLASTN, and TBLASTX. They'll come.

\paragraph{Profile/profile comparison.} A number of pioneering papers and
software packages have demonstrated the power of profile/profile
comparison for even more sensitive remote homology detection. We will
aim to develop profile HMM methods in HMMER with improved detection
power, and at HMMER3 speed. In the meantime, check out
\href{https://toolkit.tuebingen.mpg.de/\#/tools/hhblits}{HHBLITS} from
Johannes S\"oding's group.


\subsection{What's coming}

HMMER4 has been under development\footnote{Excruciatingly \emph{slow}
  development. Hey, it's tough being a PI and trying to keep your own
  research project.} since 2011. Some of the stuff we're working on:

\paragraph{The return of glocal alignment.} Slow old HMMER2 was
 capable of ``glocal'' alignment, in which it would align a complete
 profile to a subsequence of a target sequence; this was great for
 annotating domain structure of protein sequences, among other things.
 In developing our speed heuristic for HMMER3, for numerical reasons,
 we had to sacrifice glocal alignment; HMMER3 \emph{only} does local
alignment. In HMMER4, we've solved the problems that prevented H3 from
using glocal alignment. H4 uses a new \emph{dual-mode} profile
architecture, combining local and glocal alignment modes in a single
probability model.

\paragraph{Memory efficiency.} The HMMER ensemble alignment
algorithms (the HMM Forward and Backward algorithms) are expensive in
memory. For most uses, you don't notice, but there are extreme cases
where you may. H3 can require as much as $\sim 36L^2$ bytes of memory
for a query sequence of length $L$, and for a 35Kaa titin sequence,
that's 44GB of RAM. In HMMER4, we've solved this with a variety of old
and new techniques.

\paragraph{Ensemble calculations everywhere.} HMMER uses ensemble
calculations (i.e., integrates over alignment uncertainty) for scoring
homologous sequences and for calculating the confidence in individual
aligned residues. However, when it decides how many domains are in a
sequence, and where they are, it uses an \emph{ad hoc} procedure that
uses ensemble information, but is not well defined. In HMMER4, we've
solved that problem with a new domain definition algorithm.

\paragraph{More processor support.} One of the attractive features of the
HMMER ``MSV'' acceleration algorithm is that it is a very tight and
efficient piece of code. The bad news is, it's a very tight and
efficient piece of \emph{assembly} code. We have to write
processor-specific SIMD vector instructions for each platform that
HMMER runs on. HMMER currently only supports x86 (Intel/AMD) and
PowerPC platforms (big-endian AIX PowerPC's, not the newer crop of
little-endian Linux PowerPC's). HMMER4 will also include support for
Linux/PowerPC and ARM NEON. It also can use the latest x86 vector
instructions (AVX and AVX-512).

\paragraph{Better parallelization.} HMMER is so fast that it's often
input-bound, rather than CPU-bound: that is, it takes longer just to
get the sequences from your disk than it takes to compare them to a
profile. That's been taxing the simple parallelization methods we
use. HMMER's multithreaded parallelization really doesn't scale well
beyond 2-4 processors, on most machines; and possibly worse, if you're
on a slow filesystem (for example, if you're reading data from a
network filesystem instead of from local disk). In H4, we're working
on improving our parallelization and our data input.

\subsection{How to learn more about profile HMMs}

\textbf{Cryptogenomicon} (\url{http://cryptogenomicon.org/}) is a blog
where we will be talking about issues as they arise in HMMER, and
where you can comment or follow the discussion.

Reviews of the profile HMM literature have been written by us
\citep{Eddy96,Eddy98} and by Anders Krogh \citep{Krogh98}.

For details on how profile HMMs and probabilistic models are used in
computational biology, see the pioneering 1994 paper from Krogh et
al. \citep{Krogh94} or our book \emph{Biological Sequence Analysis:
Probabilistic Models of Proteins and Nucleic Acids} \citep{Durbin98}.

For more details on many aspects of the inner workings of HMMER3, see
\citep{Eddy08,Eddy09b,Eddy11}.

To learn more about HMMs from the perspective of the speech
recognition community, an excellent tutorial introduction has been
written by Rabiner \citep{Rabiner89}.

\vspace{2em}
\begin{srefaq}{How do I cite HMMER?}
There has never been a paper on the HMMER software, and it looks like
there may never be. If you're writing
for an enlightened (url-friendly) journal, the best reference is
\url{http://hmmer.org/}.  If you must use a paper reference, currently
the best one to use is the 2011 paper on HMMER3 acceleration \citep{Eddy11}.
\end{srefaq}















  









