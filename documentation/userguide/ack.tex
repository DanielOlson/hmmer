\section{Acknowledgements and history}

HMMER 1 was developed at the MRC Laboratory of Molecular Biology,
Cambridge UK, while I was a postdoc with Dr. Richard Durbin. I thank
the Human Frontier Science Program and the National Institutes of
Health for their support. 

HMMER 1.8 (and subsequent minor releases) was the first public release
of HMMER in April 1995. A number of modifications and improvements
went into HMMER 1.9 code, but 1.9 was never released. Some versions of
HMMER 1.9 did inadvertently escape St. Louis and make it to other
sites, but it was never documented or supported. HMMER 1.9 collapsed
under its own weight in 1996.

HMMER 2 is a nearly complete rewrite, based on a new model
architecture dubbed ``Plan 7''. Implementation was begun in November
1996 at Washington University in St. Louis. I thank the Washington
University Dept. of Genetics, the NIH National Human Genome Research
Institute, and Monsanto for their support during this time, and to the
Biochemistry Academic Contacts Committee at Eli Lilly \& Co. for a
gift that paid for the trusty Linux laptop on which much of HMMER 2
was written... much of it in coffee bars, airport lounges, and
transoceanic flights in various parts of the world.

HMMER has now settled into a comfortable middle age, like its author -
still actively maintained, but no dramatic new changes planned.  HMMER
2.1.1 was the stable release for three years, from 1998-2001.  HMMER
2.2g was the stable release for two more years, 2001-2003. The latest
release, 2.3, was assembled in spring 2003.

The MRC-LMB computational molecular biology discussion group
contributed many ideas to HMMER. In particular, I thank Richard
Durbin, Graeme Mitchison, Erik Sonnhammer, Alex Bateman, Ewan Birney,
Gos Micklem, Tim Hubbard, Roger Sewall, David MacKay, and Cyrus
Chothia. 

Sequence format parsing ({\tt sqio.c}) in HMMER is derived from an
early release of the {\tt READSEQ} package by Don Gilbert, Indiana
University. Thanks to Don for an excellent piece of software; and
apologies for the mangling I've put it through since.  The file {\tt
hsregex.c} is a derivative of Henry Spencer's regular expression
library; thanks, Henry. Several miscellaneous functions in {\tt
sre\_math.c} are taken from public domain sources and are credited in
the code's comments. {\tt masks.c} includes a modified copy of the XNU
source code from David States and Jean-Michel Claverie. The Altivec
port to Macintosh PowerPC was contributed by Erik Lindahl at Stanford.

In many other places, I've reimplemented algorithms described in the
literature. These are too numerous to credit and thank here. The
original references are given in the comments of the code. However,
I've borrowed more than once from the following folks that I'd like to
be sure to thank: Steve Altschul, Pierre Baldi, Phillip Bucher, Warren
Gish, David Haussler, Steve and Jorja Henikoff, Richard Hughey, Kevin
Karplus, Anders Krogh, Bill Pearson, and Kimmen Sj\"{o}lander.

HMMER is primarily developed on GNU/Linux machines, but is tested on a
variety of hardware. I thank Compaq, IBM, Intel, Sun Microsystems,
Silicon Graphics, Hewlett-Packard, and Paracel for generous hardware
support. Dave Cortesi at SGI contributed much useful advice on the
POSIX threads implementation. I owe a tremendous debt to the free
software community for the development tools I use: an incomplete list
includes GNU gcc, gdb, emacs, and autoconf; Cygnus' and others' egcs
compiler; Conor Cahill's dbmalloc library; Bruce Perens'
ElectricFence; Armin Biere's ccmalloc; the cast of thousands that
develops CVS, the Concurrent Versioning System; Larry Wall's perl;
LaTeX and TeX from Leslie Lamport and Don Knuth; Nikos Drakos'
latex2html; Thomas Phelps' PolyglotMan; Linus Torvalds' Linux
operating system; and the folks at Red Hat Linux and Mandrake Linux.

Finally, I will cryptically thank Dave ``Mr. Frog'' Pare and Tom
``Chainsaw'' Ruschak for a totally unrelated free software product
called Empire that was historically instrumental in HMMER's
development -- for reasons that are best not discussed while sober.

