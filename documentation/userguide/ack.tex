\section{Acknowledgements and history}

HMMER1 was developed on slow weekends in the lab at the MRC Laboratory
of Molecular Biology, Cambridge UK, while I was a postdoc with Richard
Durbin and John Sulston, supposedly working on the molecular and
developmental genetics of axon guidance in \emph{C. elegans}.  I thank
the Human Frontier Science Program and the National Institutes of
Health for their remarkably enlightened and tolerant support of my
postdoctoral research.

The MRC-LMB computational molecular biology discussion group
contributed many ideas to HMMER. In particular, I thank Richard
Durbin, Graeme Mitchison, Erik Sonnhammer, Alex Bateman, Ewan Birney,
Gos Micklem, Tim Hubbard, Roger Sewall, David MacKay, and Cyrus
Chothia. 

The UC Santa Cruz HMM group, led by David Haussler and including
Richard Hughey, Kevin Karplus, Anders Krogh (now back in Copenhagen)
and Kimmen Sj\"{o}lander, has been a source of knowledge, friendly
competition, and occasional collaboration. All scientific competitors
should be so gracious. The Santa Cruz folks have never complained (at
least in my earshot) that HMMER started as simply a re-implementation
of their original ideas, just to teach myself what HMMs were.

HMMER 1.8, the first public release of HMMER, came in April 1995,
shortly after I moved to Washington University in St. Louis. A few
bugfix releases followed. A number of more serious modifications and
improvements went into HMMER 1.9 code, but 1.9 was never
released. Some versions of HMMER 1.9 did inadvertently escape
St. Louis and make it to some genome centers, but 1.9 was never
documented or supported. The HMMER 1.9 development line collapsed
under its own weight in 1996.

A nearly complete rewrite was begun in November 1996, based on the new
Plan 7 model architecture, and this became HMMER2. I thank the
Washington University Dept. of Genetics, the NIH National Human Genome
Research Institute, and Monsanto for their support during this time.
Also, I thank the Biochemistry Academic Contacts Committee at Eli
Lilly \& Co. for a gift that paid for the trusty and indispensable
Linux laptop on which much of HMMER 2 was written. Much of HMMER was
written in coffee shops, airport lounges, transoceanic flights, and
Graeme Mitchison's kitchen. The source code still contains a
disjointed record of where and when various bits were written.

Sequence format parsing in HMMER was derived from an early release of
the {\tt READSEQ} package by Don Gilbert, Indiana University. Thanks
to Don for an excellent piece of software, and apologies for the years
of mangling I've put it through since I obtained it in 1992. My
regular expression code was derived from Henry Spencer's regular
expression library; thanks, Henry. Several other miscellaneous
functions are taken from public domain sources and are credited in the
code's comments.

John Blanchard (Incyte Pharmaceuticals) made several contributions to
HMMER2's PVM implementation, and the remaining bugs were my own dumb
fault.  Dave Cortesi (Silicon Graphics) contributed much useful advice
on HMMER2's POSIX threads implementation. A blazing fast Altivec port
to Macintosh PowerPC was contributed in toto by Erik Lindahl at
Stanford. In many other places, I reimplemented algorithms described
in the literature, too numerous to credit and thank here. The original
references are given in the comments of the code. However, I've
borrowed more than once from the following folks that I'd like to be
sure to thank: Steve Altschul, Pierre Baldi, Phillip Bucher, Warren
Gish, Steve and Jorja Henikoff, Anders Krogh, and Bill Pearson.

HMMER then settled into a comfortable middle age, like its author;
actively maintained, but dramatic changes seemed increasingly
unlikely. HMMER 2.1.1 was the stable release for three years, from
1998-2001.  HMMER 2.2g was intended to be a beta release, but became
the \emph{de facto} stable release for two more years, 2001-2003. The
2.3 release series was assembled in spring 2003. HMMER 2.3.2 became
the longest running stable release, from 2003-2008.

But much as Beowulf rose from his comfortable middle age to fight one
last dragon, development of a third version of HMMER began in December
2004.

To my mind, the fundamental methods for using probability theory in
biological sequence comparison were established in the 1990's
\citep{Durbin98}. But aside from our own implementations (HMMER for
proteins, Infernal for RNAs), probabilistic methods were not widely
adopted by widely used sequence homology search tools, despite being
highly influential in the foundation of other sequence analysis
methods such as genefinders.  In particular, it seemed increasingly
clear that BLAST, the community's most widely used database search
tool, was unlikely to adopt a probabilistic framework. Perhaps showing
middle-aged conceit for one's legacy, this was gnawing at me. I
decided to take up arms one last time, and build an implementation
that could go head to head with BLAST.

The crux of the matter was simple: speed. BLAST was 100-fold faster
than HMMER, and that was clearly the main reason why BLAST was useful
and HMMER was relegated to the periphery. Could HMMER be accelerated?
Even better, could it be accelerated so much that we could afford to
use the more powerful (but slower) HMM Forward and Backward
algorithms, for full probabilistic inference integrated over alignment
uncertainty, obviating the need to score only a single optimal
alignment? The key acceleration algorithms in HMMER3 came together
suddenly, in two winter months of 2007-2008, as the result of reading
two key papers: Michael Farrar's 2007 paper on SIMD vector
parallelization of Smith/Waterman \citep{Farrar07} and Torbj{\o}rn
Rognes' 2001 paper on a heuristic search algorithm called PARALIGN
\citep{Rognes01}. The rest of the code came together far more slowly,
as the entire package was once again rewritten nearly from scratch,
this time built to do full probabilistic inference from the ground up,
in the newly accelerated framework.

The result is HMMER3. I hope it will serve as a foundation for much
future work in applying probabilistic inference to remote sequence
similarity recognition and alignment applications. Whether it suffices
to defeat the dragon -- we will see.


HMMER is primarily developed on GNU/Linux machines, but is tested on a
variety of hardware. At various times over the past two decades,
Compaq, IBM, Intel, Sun Microsystems, Silicon Graphics,
Hewlett-Packard, and Paracel have provided the generous hardware
support that makes all this possible. I thank the many engineers from
IBM, Sun, Silicon Graphics, Microsoft, nVidia, DEC, Compaq, MasPar,
and other companies living and dead who have made many suggestions for
improvements, many of which I used, and some of which I understood. I
owe a large debt to the free software community for the many
development tools I use: an incomplete list includes GNU gcc, gdb,
emacs, and autoconf; Julian Seward's remarkable Valgrind; the
Subversion version control system; Perl; \LaTeX; Nikos Drakos'
latex2html; Thomas Phelps' PolyglotMan; and of course Linus Torvalds'
Linux operating system.

Finally, I will cryptically thank Dave ``Mr. Frog'' Pare and Tom
``Chainsaw'' Ruschak for a totally unrelated free software product
that was historically instrumental in HMMER's development -- for
reasons that are best not discussed while sober.

\label{manualend}
